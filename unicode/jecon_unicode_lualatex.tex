%#!lualatex
%#BIBTEX upbibtex jecon_unicode_lualatex.aux
% 
% このファイルは lualatex でコンパイルすることを前提で作成しています。
%
%############################## Main #################################

%% 以下は lualatex 用の設定。
\documentclass{ltjsarticle}

%% 以下は lualatex 用の設定。
\usepackage{fontspec}
\usepackage{luatexja-fontspec}

%% 以下は共通の設定
\usepackage{natbib}
\usepackage{url}
\usepackage{amsmath,amssymb,amsthm}

%% 色を付ける.
\usepackage{graphicx}
\usepackage{color}
\definecolor{MyBrown}{rgb}{0.3,0,0}
\definecolor{MyBlue}{rgb}{0,0,0.3}
\definecolor{MyRed}{rgb}{0.6,0,0.1}
\definecolor{MyGreen}{rgb}{0,0.4,0}
\usepackage[bookmarks=true,%
bookmarksnumbered=true,%
colorlinks=true,%
linkcolor=MyBlue,%
citecolor=MyRed,%
filecolor=MyBlue,%
pagecolor=MyBlue,%
urlcolor=MyGreen%
]{hyperref}

%% \BibTeX command を定義.
\makeatletter
\def\BibTeX{{\rm B\kern-.05em{\sc i\kern-.025em b}\kern-.08em
    T\kern-.1667em\lower.7ex\hbox{E}\kern-.125emX}}
\makeatother

%%% title, author, acknowledgement, and date
\title{\textbf{\texttt{jecon.bst} とユニコード文字}}

\author{武田史郎\thanks{Email: \texttt{\href{mailto:shiro.takeda@gmail.com}{shiro.takeda@gmail.com}}}}

% \date{平成28年2月3日}
\date{\today}

%#####################################################################
%######################### Document Starts ###########################
%#####################################################################
\begin{document}

\begin{flushleft}
 {\Large \textbf{LuaLaTeX (\texttt{lualatex}) を利用するケース}}
\end{flushleft}

\vspace{1em}

\section{使い方}

\subsection{TeX ファイルの書き方}

\begin{itemize}
 \item このファイル (\texttt{jecon\_unicode\_lualatex.tex}) を参考にしてください。
 \item \texttt{bib} ファイルの書き方、\texttt{jecon.bst} の使い方は
       \texttt{jecon\_unicode\_xelatex.pdf} を見てください。
\end{itemize}

\subsection{コンパイル}

\begin{itemize}
 \item LuaLaTeX でのコンパイル。
\begin{verbatim}
lualatex jecon_unicode_lualatex.tex
upbibtex jecon_unicode_lualatex
lualatex jecon_unicode_lualatex.tex               
lualatex jecon_unicode_lualatex.tex
\end{verbatim}
 \item つまり、\TeX のコマンドとしては \texttt{lualatex} を、BibTeX のコマンドと
       しては \texttt{upbibtex} を利用します。
\end{itemize}

\section{例}

\input{unicode_example.tex}

\nocite{*}

%%% BibTeX スタイルファイルの指定.jecon_unicode.bst を指定.
\bibliographystyle{jecon_unicode}

%% BibTeX データベースファイルの指定.
%
\bibliography{jecon_example_reverse,unicode_example}

\end{document}
%#####################################################################
%######################### Document Ends #############################
%#####################################################################

% --------------------
% Local Variables:
% fill-column: 80
% coding: utf-8-dos
% End:


