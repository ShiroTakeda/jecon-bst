%#!lualatex
%#BIBTEX upbibtex jecon-number

% Author:	       Shiro Takeda
% First-written:       <2002/11/02>
%
%############################## Main #################################

%% jlreq クラスを利用
\documentclass[article]{jlreq}

%% 以下は lualatex 用の設定。
\usepackage[no-math]{fontspec}
\usepackage{luatexja-fontspec}
\usepackage{luatexja-preset}

%% natbib.sty を使う。
\usepackage[numbers]{natbib}
\usepackage{newtxtext,newtxmath}
\usepackage{color}
\definecolor{MyBrown}{rgb}{0.3,0,0}
\definecolor{MyBlue}{rgb}{0,0,0.3}
\definecolor{MyRed}{rgb}{0.6,0,0.1}
\definecolor{MyGreen}{rgb}{0,0.4,0}
\usepackage[%
bookmarks=true,%
bookmarksnumbered=true,%
colorlinks=true,%
linkcolor=MyBlue,%
citecolor=MyRed,%
filecolor=MyBlue,%
urlcolor=MyGreen%
]{hyperref}

% \cicet 命令が \citep 命令と同じになるように再定義。
\renewcommand{\citet}[1]{\citep{#1}}

% 番号の後に少し空白を入れる。
\renewcommand{\bibnumfmt}[1]{[#1]\hspace{0.5em}}

%#####################################################################
%######################### Document Starts ###########################
%#####################################################################
\begin{document}

\verb|natbib.sty|の番号モードにより、番号での引用をおこなっているもの。
\vspace*{1em}

\texttt{natbib.sty} の番号モードを利用するため次のような設定にしている。
\begin{itemize}
 \item まず、\texttt{natbib.sty} を次のように ``\texttt{numbers}''オプション付き
       で読み込んでいる\\
       \verb|\usepackage[numbers]{natbib}|
 \item bst ファイルには ``\verb|jecon-no-sort.bst|'' を利用している。この bstファ
       イルではレファレンスは引用順に並べられる。
 \item \verb|\citet| 命令の代わりに \verb|\citep| 命令で引用している\\
       注:厳密には、\texttt{citep} 命令と同じ機能になるように再定義した \texttt{citet} 命令を利用している。
\end{itemize}

\vspace{1em}


引用部分:
\citet{40020418914},
\citet{yamazaki13:_japan},
\citet{takeda2013jecon},
\citet{Takeda2012a},
\citet{arimura-takeda2012},
\citet{matloff__2012},
\citet{Boswell-2012},
\citet{takeda2012_cge},
\citet{Takeda2014a},
\citet{takeda2019a},
\citet{40018847518},
\citet{takeda10:_cge_analy_welfar_effec_trade},
\citet{40017004376},
\citet{2009yamasue502165},
\citet{Biker-2007-unemployment},
\citet{takeda06:_cge_analy_welfar_effec_trade},
\citet{2007yamasue482353},
\citet{BabikerRutherford-2005-EconomicEffectsof},
\citet{somusho04jp:2000io-kaisetsu},
\citet{ishikawa03:_green_gas_emiss_contr_open_econom},
\citet{brooke03:_gams},
\citet{Hattori02},
\citet{miyazawa02:_io_intr},
\citet{isikawa02jp:_env_trade},
\citet{Hattori01},
\citet{katayama2001},
\citet{Babiker2000525},
\citet{rutherford00:_gtapin_gtap_eg},
\citet{Hattori00},
\citet{a.___2000},
\citet{fujita99jp:_spatial_econom},
\citet{Babiker-1999-KyotoProtocoland},
\citet{markusen99jp:trade_vol_1},
\citet{oyama99:_mark_stru},
\citet{kuroda97jp:keo},
\citet{barro97jp},
\citet{wong95:_inter_trade_goods_factor_mobil_},
\citet{ishikawa94:_revis_stolp_samuel_rybcz_theor_produc_exter},
\citet{brezis93:_leapf_inter_compet},
\citet{kiyono93:_regu_comp_1},
\citet{krugman91:_geogr_trade},
\citet{helpman91:_inter_trade_trade_polic},
\citet{krugman91:_is_bilat_bad},
\citet{iwamoto91jp:haito-keika},
\citet{nishimura90:_micr_econ},
\citet{wang89:_model_therm_hydrod_aspec_molten},
\citet{ito85:_inte_trad},
\citet{lucas76:_econom_polic_evaluat},
\citet{imai72:_micr_2},
\citet{imai71:_micr_1},
\citet{milne-thomson68:_theor_hydrod},
\citet{Ryza2016},
\citet{ThoughtWorksinc.08},
\citet{Ryza15:_advan_analy_spark_patter_learn_data_scale},
\citet{naikakufu_2011},
\citet{takeda-gtap-2016ja},
\citet{Parry1997},
\citet{DeGorter2002},
\citet{Peri2007},
\citet{takeda07jp:CGE_for_rieti},
\citet{hosoda2017},
\citet{takeda2017300208},
\citet{Attwood06:SexedUp_art},
\citet{Attwood09:Mainstreaming},
\citet{attwood2010porn},
\citet{jones84:_handb_inter_econom},
\citet{jones85:_handb_inter_econom},
\citet{jones97:_handb_inter_econom},
\citet{arimura-katayama-matsumoto-2017jp},
\citet{doi:10.1175/2009BAMS2778.1},
\citet{doi:10.1175/2009BAMS2778.1},
\citet{Li_2018},
\citet{romer19jp:_advan_macroecon}

% Local Variables:
% fill-column: 80
% coding: utf-8-dos
% End:



引用部分:
\citet{sample-test2},
\citet{sample-test},
\citet{oo100:_no_title},
\citet{oo99:_no_title},
\citet{120005678435},
\citet{120005614155},
\citet{Krey2014},
\citet{有村-蓬田2012},
\citet{森201206},
\citet{Jaeger2011},
\citet{Ades-2010-EnergyUseand},
\citet{松浦2010a},
\citet{bohringer2007measuring},
\citet{Ahman2007},
\citet{横溝2007},
\citet{Bohringer2006},
\citet{chuokankyo06jp:ccs},
\citet{瀧川2006a},
\citet{bouet06:_is_erosion_of_tarif_prefer_serious_concer},
\citet{Mcconnell2005},
\citet{Stokey2004},
\citet{Loschel2002},
\citet{ihori02:_japa_tax},
\citet{hardle_sweave:_2002},
\citet{nakamura00:_excel_io},
\citet{Iregui-1999-EFFICIENCYGAINSFROM},
\citet{okura96jp:nihon-no-zeisei},
\citet{MaggiRodr'iguez-Clare-1998-ValueofTrade},
\citet{shigen97:_energ_balan},
\citet{内田90},
\citet{samuelson67:_econo_found},
\citet{allais1953},

% Local Variables:
% fill-column: 80
% coding: utf-8-dos
% End:


\vspace{1em}

このようなまとめての引用も可能。 → \citet{40020418914, yamazaki13:_japan,
takeda2013jecon, Takeda2012a, takeda2012_cge, Takeda2014a, takeda2019a, 40018847518}

\nocite{*}

%% BibTeX スタイルファイルの指定。
%
% jecon-no-sort.bst を指定。
\bibliographystyle{jecon-no-sort}

%% BibTeX データベースファイルの指定。
%
% 一個上のフォルダにある jecon-example.bib, jecon-example-unicode.bib を指定。
\bibliography{../jecon-example,../jecon-example-unicode}

\end{document}

%#####################################################################
%######################### Document Ends #############################
%#####################################################################
% </pre></body></html>
% --------------------
% Local Variables:
% fill-column: 80
% End:
