%#!lualatex
%#BIBTEX upbibtex jecon-reverse

% Author:	       Shiro Takeda
% First-written:       <2014/01/23>
%
%############################## Main #################################

%% jlreq クラスを利用
\documentclass[article]{jlreq}

%% 以下は lualatex 用の設定。
\usepackage[no-math]{fontspec}
\usepackage{luatexja-fontspec}
\usepackage{luatexja-preset}

%% natbib.sty を使う。
\usepackage{natbib}
\usepackage{newtxtext,newtxmath}
\usepackage{color}
\definecolor{MyBrown}{rgb}{0.3,0,0}
\definecolor{MyBlue}{rgb}{0,0,0.3}
\definecolor{MyRed}{rgb}{0.6,0,0.1}
\definecolor{MyGreen}{rgb}{0,0.4,0}
\usepackage[%
bookmarks=true,%
bookmarksnumbered=true,%
colorlinks=true,%
linkcolor=MyBlue,%
citecolor=MyRed,%
filecolor=MyBlue,%
urlcolor=MyGreen%
]{hyperref}

%#####################################################################
%######################### Document Starts ###########################
%#####################################################################
\begin{document}

\verb|\bibliographystyle|に\verb|jecon-reverse.bst|を指定し、
\verb|\bibliography|に\verb|jecon-example-old.bib|を指定。
\vspace{1em}\\

普通、\verb|bib|ファイルでは日本人名は「姓 名」という形で名前を指定するが、
\verb|jecon.bst| では日本人名についても外国人名と同じように「姓, 名」という形式で
指定する必要がある。ただし、これまでの伝統的な記述方法も使えるようになっている。
\verb|jecon-example-old.bib|では通常の bib ファイルのように「姓 名」という形式で
著者名が記述されている。そのような \verb|bib|ファイルでも適切扱う設定にしている
のが \verb|jecon-reverse.bst|。
\begin{itemize}
 \item 注: \verb|bst.sei.mei.order| という関数の中身を書き換えている以外は
       \verb|jecon.bst| と同じです。
\end{itemize}

\vspace{2em}

\input{cited-part}

\nocite{*}

%% BibTeX スタイルファイルの指定。
\bibliographystyle{jecon-reverse}
% \bibliographystyle{jecon-new}

%% BibTeX データベースファイルの指定。
%
% 一個上のフォルダにある jecon-example-old.bib を指定。
\bibliography{../jecon-example-old}

\end{document}

%#####################################################################
%######################### Document Ends #############################
%#####################################################################
% </pre></body></html>
% --------------------
% Local Variables:
% fill-column: 80
% End:

